\documentclass{beamer}
%\usetheme{Rochester}
\title[Integer Factorisation with Elliptic Curves]{Integer Factorisation with Elliptic Curves}
\author{Will Bolton}
\date{\today}

\begin{document}
\titlepage
\begin{frame} % SLIDE 1
\frametitle{Fields and finite fields}
\begin{definition}
	A field is a commutative, unital ring in which every non-zero element is invertible
\end{definition}
\vfill
	The group $Z_n$ is a field if and only if $n$ is prime.
\end{frame}

\begin{frame} % SLIDE 2
\frametitle{The Euclidean Algorithm}
\begin{definition}
	The euclidean algorithm takes two numbers and returns their \emph{greatest common divisor (gcd)} by repeated division with remainder.
\end{definition}
\begin{align*}
	\gcd{21,15}:21 &= 1\times15 + 6\\
	15 &= 2\times6 + 3\\
	6 &= 2\times3 + 0\\
\end{align*}
So $\gcd(21,15)=3$
\end{frame}

\begin{frame} % SLIDE 3
\frametitle{Elliptic curves and the projective plane}
\begin{definition}
	The projective plane is an extension of regular 2-dimensional euclidean space by adding ``points at infinity'' such that every pair of lines intersects exactly once.
\end{definition}
\begin{definition}
	An elliptic curve is a non-singular cubic projective curve. For our purposes, they can all be written as $y^2 = f(x)$, where $f(x)$ is a cubic polynomial in $x$ with no repeated roots.
\end{definition}
\end{frame}

\begin{frame} % SLIDE 4
\frametitle{The elliptic curve addition law}
The points on an elliptic curve can be turned into a group via the ``chord-tangent law''
\end{frame}

\begin{frame} % SLIDE 5
\frametitle{Lenstra's algorithm}
\begin{definition}
	Lenstra's algorithm goes as follows to factor an integer $N$:
\end{definition}
\end{frame}

\begin{frame} % SLIDE 6
\frametitle{Example}

\end{frame}

\end{document}
